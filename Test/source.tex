\documentclass[9pt]{article}
\usepackage[utf8]{inputenc}
\usepackage[T1]{fontenc}
\usepackage{ngerman}
\usepackage{geometry}
\usepackage{tabularx}
\newcolumntype{Z}{>{\centering\arraybackslash}X}
\newcolumntype{Y}{>{\hsize=0.7\hsize}Z}
\newcolumntype{T}{>{\hsize=0.3\hsize}Z}
\newcolumntype{F}{>{\hsize=0.2\hsize}X}
\renewcommand{\thesection}{\arabic{section}.\qquad}
\renewcommand{\thesubsection}{\alph{subsection})}
\renewcommand{\thesubsubsection}{\arabic{section}-\alph{subsection}-\arabic{subsubsection})}
\geometry{
  top=5mm,
  left=5mm,
  right=5mm,
}
\begin{document}
\large{
\begin{center}
  \begin{tabularx}{\linewidth}{|Z|}
    \hline
    \textbf{Test 03 - Deadlocks und Scheduling - Betriebssysteme}\\%HEADER
    \hline
        {
          \begin{tabularx}{\linewidth}{Y|T|Y|T}
            \textbf{Erstellende Stammgruppe} & 11 & \textbf{Lehrverband} & 2B2\\
          \end{tabularx}
        }\\%INFOS ZU GRUPPE
        \hline
            {
              \begin{tabularx}{\linewidth}{F|lr}
                \textbf{Gruppenmitglieder} & Ovidiu - Dan Bogat & ic17b501\\
                & Barbara Eder & ic17b066\\
                & Ferhat Doevme & ic15b046\\
              \end{tabularx}
            }\\
            \hline
            {
              \begin{tabularx}{\linewidth}{F|}
                \textbf{Name}\\
              \end{tabularx}
            }\\
            \hline
            {
              \begin{tabularx}{\linewidth}{F|}
                \textbf{Matrikelnummer}\\
              \end{tabularx}
            }\\
            \hline
            {
              \begin{tabularx}{\linewidth}{F|}
                \textbf{Lehverband}\\
              \end{tabularx}
            }\\
            \hline
            {
              \begin{tabularx}{\linewidth}{F|}
                \textbf{Punkte}\\
              \end{tabularx}
            }\\
            \hline
            
  \end{tabularx}
\end{center}
}
\bigskip
\begin{center}
  \begin{minipage}{0.9\linewidth}
    \section{Deadlock}
    \subsection{Erklären Sie den unterschied zwischen einer \textsl{unterbrechbaren} und einer \textsl{nicht unterbrechbaren Ressource} ()}
	\vspace{25mm}	
    \subsection{Erklären Sie einen \textsl{Deadlock} in eigenen Worten (12 Punkte)}
\vspace{65mm}
    \subsection{Bei welchen Arten von \textsl{Ressourcen} kann ein \textsl{Deadlock} entstehen? ( Punkte)}
\end{minipage}
\newpage
  \begin{minipage}{0.9\linewidth}
    \subsection{Welche 4 Bedingungen müssen gleichzeitig erfüllt sein, damit ein Ressourcen-Deadlock überhaupt entstehen kann? ( Punkte)}
    \vspace{25mm}
    \subsubsection{Nennen Sie 2 der insgesamt 4 Strategien mit denen ein bestehender Deadlock behandelt werden kann ( Punkte)}
    \vspace{15mm}
    \subsubsection{Nennen Sie 3 Arten der Behebung von einem Ressourcen-Deadlock ( Punkte)}
  \end{minipage}
\end{center}
\vspace{25mm}
\begin{center}
  \begin{minipage}{0.9\linewidth}
    \section{Scheduling}
    \subsection{In welche 2 Gruppen werden Prozesse bezüglich Scheduling gegliedert? ( Punkte)}
    \vspace{10mm}
    \subsection{Nennen Sie alle 3 Kategorien von Schedulingstrategien und geben Sie jeweils ein Beispiel aus einer Kategorie ( Punkte)}
    
  \end{minipage}
\end{center}

\end{document}
